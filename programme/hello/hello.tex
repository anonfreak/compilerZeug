%%%%%%%%%%%%%%%%%%%%%%%%%%%%%%%%%%%%%%%%%%%%%%%%%%%%%%%%%%%%%%%%%%%%%%%%%%%%%%%
%% Project:     Vorlesungen an der Berufsakademie Karlsruhe
%% Descr:       Programmieren / Das Hello-World Programm
%% Author:      Dr. J�rgen Vollmer, Juergen.Vollmer@informatik-vollmer.de
%% $Id: hello.tex,v 1.3 2007/06/01 10:38:31 vollmer draft $
%%%%%%%%%%%%%%%%%%%%%%%%%%%%%%%%%%%%%%%%%%%%%%%%%%%%%%%%%%%%%%%%%%%%%%%%%%%%%%%

\documentclass[german%
  %,article
  ,landscape
  ,slidesonly%  Try notes or notesonly instead.
  %,notes%       Use instead of slidesonly to typeset the notes.
  %,notesonly%   Use instead of slidesonly to typeset notes and slides.
  %,semcolor%    Try me if using PSTricks.
  %,semrot%      Try me if using Rokicki's dvips.
  ,slidesec%
  ,semhelv%      Try me if using a PostScript printer.
  ,sem-a4
]{seminar}

\usepackage{bibgerm}
\usepackage{folien}

%%%%%%%%%%%%%%%%%%%%%%%%%%%%%%%%%%%%%%%%%%%%%%%%%%%%%%%%%%%%%%%%%%%%%%%%%%%%%%%

\begin{document}
\rcsInfo $Id: hello.tex,v 1.3 2007/06/01 10:38:31 vollmer draft $

\begin{allversions}
  \Section{Programmieren}
\end{allversions}
%%==================

\begin{slide}
\slideheading{.c-Dateien -- hello.c}

\medskip
\renewcommand{\baselinestretch}{0.8}
\footnotesize

\begin{alltt}
# include <stdio.h>
# include "hello2.h"

int main (int argc, char *argv[])
\{
  printf ("Hello World\verb+\+n");
  printf ("Die Kommandozeilenargumente:\verb+\+n");

  for (i = 0; i < argc; i++) \{
    hello (i, argv[i]);
  \}

  return 0;
\}
\end{alltt}

\vfill
\end{slide}

%%%%%%%%%%%%%%%%%%%%%%%%%%%%%%%%%%%%%%%%%%%%%%%%%%%%%%%%%%%

\begin{slide}
\slideheading{.h-Dateien -- hello2.h}

\medskip
\renewcommand{\baselinestretch}{0.8}
\footnotesize
\begin{alltt}

# ifndef hello2_H
# define hello2_H
# include <stdio.h>

typedef int Integer;
extern void hello (Integer nr, char *argument);
extern  Integer i;

# ifdef DEBUG
# define DPRINTF(str)                          \verb+\+
  fprintf (stderr, "Datei %s, Zeile %d: %s\verb+\+n", \verb+\+
                   __FILE__, __LINE__, str);
  /* __FILE__ und __LINE__ sind vordefinierte Macros, die
   * den Namen der �bersetzten Datei und die Zeilennumer
   * an der das Macro benutzt wird enthalten.
   */
# else
# define DPRINTF(str)  /* ignored */
# endif
# endif
\end{alltt}

\vfill
\end{slide}

%%%%%%%%%%%%%%%%%%%%%%%%%%%%%%%%%%%%%%%%%%%%%%%%%%%%%%%%%%%

\begin{slide}
\slideheading{.c-Dateien -- hello2.c}{}

\medskip
\renewcommand{\baselinestretch}{0.8}
\footnotesize

\begin{alltt}
# include <stdio.h>
# include "hello2.h"

Integer i = 0;

void hello (Integer nr, char *argument)
\{
  DPRINTF ("Aufruf von hello");
  printf (" - %2d: %s\verb+\+n", nr, argument);
\}
\end{alltt}

\vfill
\end{slide}

%%%%%%%%%%%%%%%%%%%%%%%%%%%%%%%%%%%%%%%%%%%%%%%%%%%%%%%%%%

\begin{slide}
\slideheading{Kommandos}

\medskip
\renewcommand{\baselinestretch}{0.8}
\footnotesize

\begin{tabular}{l|l}
\texttt{cc -c hello.c}
& \emph{hello.c \Ra\ hello.o} \\
\texttt{cc -c hello2.c}
& \emph{hello2.c \Ra\ hello2.o} \\
\texttt{cc -o hello hello.o hello2.o}
& \emph{hello.o hello2.o libc.a ...  \Ra\ hello} \\
\texttt{hello} &
\emph{Hello World....}\\
\end{tabular}

\vspace{1cm}

\hrule

\begin{tabular}{ll}
\texttt{cc -DDEBUG -c hello.c} &\\
\texttt{cc -DDEBUG -c hello2.c}&\\
\texttt{cc -DDEBUG -o hello hello.o hello2.o}&\\
\end{tabular}

\vspace{1cm}

\vfill
\end{slide}

%%%%%%%%%%%%%%%%%%%%%%%%%%%%%%%%%%%%%%%%%%%%%%%%%%%%%%%%%%%

\begin{slide}
\slideheading{Makefile}{}

\medskip
\renewcommand{\baselinestretch}{0.8}
\scriptsize

\begin{alltt}
SRC_H   =         hello2.h
SRC_C   = hello.c hello2.c
OBJ     = hello.o hello2.o
MAIN    = hello

CC       = gcc
X_DEBUG  = -g -Wall -pedantic
X_CFLAGS = -ansi $(X_DEBUG) $(CFLAGS)

all: $(MAIN)

$(MAIN): $(OBJ)
\Ra $(CC) $(X_CFLAGS) -o $(MAIN) $(OBJ)

.SUFFIXES: .c .o
.c.o:
\Ra $(CC) $(X_CFLAGS) -c $*.c

clean:
\Ra rm -f $(OBJ) $(MAIN) core

hello.o:  hello.c hello2.h
hello2.o: hello2.c hello2.h

\end{alltt}

Das Zeichen \Ra\ steht f�r ein Tabulator-Zeichen.

\vfill
\end{slide}

%%%%%%%%%%%%%%%%%%%%%%%%%%%%%%%%%%%%%%%%%%%%%%%%%%%%%%%%%%

\end{document}

%%%%%%%%%%%%%%%%%%%%%%%%%%%%%%%%%%%%%%%%%%%%%%%%%%%%%%%%%%%%%%%%%%%%%%%%%%%%%%%
% Local Variables:
% TeX-parse-self: t
% TeX-master: "hello"
% TeX-auto-save: f
% End:
%%%%%%%%%%%%%%%%%%%%%%%%%%%%%%%%%%%%%%%%%%%%%%%%%%%%%%%%%%%%%%%%%%%%%%%%%%%%%%%
